%%%%%%%%%%%%%%%%%%%%%%%%%%%%%%%%%%%%%%%%%%%%%%%%%%%%%%%%%%%
% Karine Perraut <karine.perraut@univ-grenoble-alpes.fr>
% Lucas Labadie <labadie@ph1.uni-koeln.de>
% Arthur Vigan <arthur.vigan@lam.fr>
% Anthony Boccaletti <anthony.boccaletti@obspm.fr>
% Daniel Rouan <daniel.rouan@obspm.fr>
%%%%%%%%%%%%%%%%%%%%%%%%%%%%%%%%%%%%%%%%%%%%%%%%%%%%%%%%%%%

\begin{titlepage}
\parindent=0pt
% www.devoloppez.com \hspace*{\stretch{1}} \LaTeX intermédiaire
% Rubrique \LaTeX\hspace*{\stretch{1}} Tutoriels

% \begin{minipage}{0.33\textwidth}
% \hspace{-1cm}
% \includegraphics[width=\textwidth]{}
% \end{minipage}
% \begin{minipage}{0.33\textwidth}
% \includegraphics[width=\textwidth]{}
% \end{minipage}
% \begin{minipage}{0.33\textwidth}
% \hspace{1cm}
% \includegraphics[width=\textwidth]{}
% \end{minipage}

% \vspace*{\stretch{1}}
% \begin{center}
% % \vspace{-3.4cm}
% \includegraphics[scale=0.2]{}
% \end{center}

\vspace*{\stretch{1}}
\begin{center}\Huge

\end{center}

\hrulefill
\begin{center}\bfseries\Huge
    Caractérisation et intégration de composants d'optique intégrée sur l'interféromètre fibré FIRST au télescope Subaru pour l'étude des protoplanètes en accrétion
\end{center}
% Characterization and deployement of Photonic Integrated Circuits for the FIRST fibered interferometer at the Subaru Telescope in the context of accreting protoplanets studies
\hrulefill

\begin{center}
      du 
\end{center}


\begin{center}\bfseries\Large
par Kevin Barjot
\end{center}

\vspace*{\stretch{1}}
\begin{flushleft}\Large

\end{flushleft}

% \vspace*{\stretch{4}}
% \begin{minipage}{0.2\textwidth}
% \includegraphics[width=0.6\textwidth]{}
% \end{minipage}
% \begin{minipage}{0.4\textwidth}
% \includegraphics[width=1.2\textwidth]{}
% \end{minipage}
% \begin{minipage}{0.4\textwidth}
% \hspace{2cm}
% \includegraphics[width=0.7\textwidth]{}
% \end{minipage}


% \vspace*{\stretch{2}}
% \begin{tikzpicture}[remember picture, overlay]
%  \begin{scope}[shift={(current page.south west)},shift={(1,1)},scale=1]
%  \shade[ball color=blue,opacity=.6] (0,0) circle (10ex);
%  \shade[ball color=blue,opacity=.8] (1.7,1) circle (5ex);
%  \shade[ball color=blue,opacity=.8] (1.5,3) circle (2ex);
%  \shade[ball color=blue,opacity=.5] (-0.5,3) circle (1ex);
%  \shade[ball color=blue,opacity=.8] (1,4) circle (1ex);
%  \shade[ball color=blue,opacity=.6] (3.5,2.5) circle (2ex);
%  \shade[ball color=blue,opacity=.8] (2.5,4.5) circle (4ex);
%  \shade[ball color=blue,opacity=.5] (3,4) circle (3ex);
%  \shade[ball color=blue,opacity=.8] (4.5,4.5) circle (3ex);
%  \shade[ball color=blue,opacity=.5] (5.1,4.7) circle (2ex);
%  \shade[ball color=blue,opacity=.8] (5,6) circle (1.5ex);
%  \shade[ball color=blue,opacity=.6] (3.5,5.5) circle (2ex);
%  \shade[ball color=blue,opacity=.8] (5,3) circle (1ex);
%  \end{scope}
%  \end{tikzpicture}
\end{titlepage}