%----------------------------------------------------------------------------------------
%	FRENCH
%----------------------------------------------------------------------------------------

\newpage
\thispagestyle{empty}
\chapter*{Résumé}

Dans le cadre de ma thèse, j'ai travaillé dans le domaine de l'imagerie haut contraste à haute résolution angulaire (HRA) pour la détection et la caractérisation de compagnon stellaire ou planétaire. Je me suis plus particulièrement intéressé au cas des protoplanètes en cours de formation qui accrètent de la matière, induisant un fort rayonnement à la longueur d'onde \ha. En plus de diminuer le contraste entre l'étoile et la protoplanète à cette longueur d'onde, facilitant sa détection, la mesure de son intensité permettrait d'apporter des contraintes sur ce phénomène à l'oeuvre dans les systèmes planétaires en formation. FIRST (Fibered Interferometer foR a Single Telescope) est un instrument installé sur le banc d'optique adaptative extrême du télescope Subaru (SCExAO) et exploite la technique de masquage et de réarrangement de pupille, qui permet d'atteindre des résolutions angulaires jusqu'à deux fois plus petite que celle du télescope. Pour cela, la pupille d'entrée de l'instrument est sous-divisée en sous-pupilles dont la lumière est injectée dans des fibres optiques monomodes qui appliquent un filtrage spatial du front d'onde supprimant ainsi les aberrations optiques à l'échelle des sous-pupilles. Sur la deuxième version de FIRST (FIRSTv2), la recombinaison de ces faisceaux est effectuée par un composant d'optique intégrée, afin d'augmenter les performances d'imagerie haut contraste. Le développement de la technologie photonique dans le visible est complexe et innovant et l'objectif de ma thèse est d'évaluer ses performances et la faisabilité de son application pour l'imagerie HRA. Cela a nécessité de continuer le développement du banc de test, aussi bien au niveau du montage optique que du logiciel de contrôle, de mettre au point une procédure spécifique d'acquisition des données et de développer un programme de traitement et d'analyse de données.

Une réplique de FIRST a été développée au \ac{LESIA}, afin de pouvoir implémenter, tester et valider les nouveaux développements de composants photoniques avant leur intégration au télescope Subaru. La recombinaison de chaque paire de sous-pupille est ainsi codée sur une ou deux sorties de la puce photonique (selon la technologie utilisée) et sont imagées sur la caméra sur quelques pixels. L'échantillonnage correct des franges d'interférences nécessite une modulation temporelle de la différence de marche. Pour cela, nous utilisons un miroir segmenté contrôlable en position, pour changer le déphasage entre les sous-pupilles. Dans un premier temps, j'ai amélioré le logiciel de contrôle de ce banc pour augmenter sa rapidité ainsi que pour permettre l'acquisition de données interférométriques nécessitant le contrôle synchronisé des différents composants pour la modulation des franges (miroir déformable, lignes à retard et caméra).

J'ai ensuite caractérisé les propriétés optiques de deux puces photoniques qui utilisent deux techniques différentes de recombinaison interférométrique de faisceaux, telles que leur transmission, la quantité de fuite du signal entre les différents guides d'onde (appelé \textit{cross-talk}), leur comportement dans les deux polarisations ainsi que le contraste instrumental. Ensuite, j'ai construit un système optique simulant une source binaire sur le banc de test, qui permet d'injecter à la fois une source avec une large bande spectrale pour simuler une étoile et une source avec une bande spectrale étroite pour simuler une exoplanète avec une raie d'émission. Cela a été crucial pour démontrer les capacités de l'instrument à détecter et mesurer le signal du compagnon. Pour mesurer ce dernier, j'ai tiré profit des propriétés de phase différentielle, qui est une grandeur auto-étalonnée. En effet, la mesure de la phase sur une large bande spectrale permet l'étalonnage des mesures de phases par le signal du continuum mettant en évidence un possible signal dans la raie d'intérêt. J'ai ainsi développé un programme de traitement et d'analyse de données permettant d'estimer la phase différentielle ainsi que la visibilité complexe et la clôture de phase et de les ajuster avec un modèle étoile-compagnon. Cela m'a permis de démontrer que FIRSTv2 pouvait détecter un compagnon de type protoplanète à une séparation équivalente à $0.7 \lambda / B$ de la source centrale, avec un contraste atteignant $0.1$.

Une des puces photoniques a été intégrée dans l'instrument FIRST installé sur le banc SCExAO, ce qui a donné lieu à la première lumière de FIRSTv2 le 10 septembre 2021. J'ai ainsi acquis des données sur ciel lors de plusieurs nuits d'observations à distance mais aussi lors d'une mission à Hawaii en février 2022. Cela a été l'occasion pour moi d'intégrer le deuxième composant pour comparaison avec le premier, ainsi que de déployer le logiciel développé au laboratoire. Les résultats du traitement de ces données sur ciel ont permis d'en apprendre plus sur la méthode d'acquisition et de traitement des données interférométriques.

En conclusion, j'ai caractérisé et étudié les performances de la technologie d'optique intégrée dans le cadre de l'imagerie interférométrique à haut contraste et haute résolution dans le visible. J'ai ainsi développé le logiciel de contrôle de FIRSTv2 en laboratoire avant de le déployer sur le banc SCExAO pour sa première lumière. Ensuite, j'ai développé le programme de traitement et d'analyse de données interférométriques pour FIRSTv2, en y incluant un ajustement des observables interférométriques par un modèle de protoplanète présentant une forte raie d'émission dans son spectre. L'objectif est de caractériser les mécanismes d'accrétion de jeunes systèmes exoplanétaires en formation. Enfin, j'ai eu l'occasion de participer à de nombreuses nuits d'observations durant lesquelles j'ai acquis des données sur des cibles simples telles que des binaires d'étoiles à faible contraste compagnon/étoile.


%----------------------------------------------------------------------------------------
%	ENGLISH
%----------------------------------------------------------------------------------------

\newpage
\thispagestyle{empty}
\chapter*{Abstract}

During my thesis, I worked in the context of high contrast and high angular resolution (HRA) imaging for the detection and the characterisation of stellar or planetary companions. I was particularly interested in the case of forming protoplanets which accrete matter, inducing a strong emission at \ha~wavelength. In addition to decreasing the contrast between the star and the protoplanet at this wavelength, facilitating its detection, the measurement of its intensity would constraints this ongoing phenomenon in protoplanetary systems. FIRST (Fibered Interferometer foR a Single Telescope) is an instrument installed on the Subaru coronagraphic extreme adaptive optics (SCExAO) testbed and exploits the pupil masking and remapping technique, which allows to reach up to twice the angular resolutiono of the telescope. To achieve this, the instrument's entrance pupil is subdivided into sub-pupils whose light is injected into single-mode optical fibres that spatially filter the wavefront, thereby suppressing optical aberrations at the sub-pupil level. In the second version of FIRST (FIRSTv2), the recombination of these beams is performed by an integrated optics component in order to increase the high contrast imaging performance. The development of photonic technology in the visible range is complex and innovative and the objective of my thesis is to evaluate its performance and the feasibility of its application for HRA imaging. In that purpose I conducted the further development of the testbed, both in terms of the optical setup and the control software, developed a specific data acquisition procedure and a data processing and analysis pipeline.

A replica of FIRST has been built at LESIA, in order to implement, test and validate new developments of photonic components before their integration at the Subaru telescope. The recombination of each pair of sub-pupils is encoded on one or two outputs of the photonic chip (depending on the technology used) and imaged on the camera on a few pixels. The correct sampling of the interference fringes requires a temporal modulation of the optical path difference (OPD). For this purpose, we use a segmented mirror controllable in piston and tip/tilt to change the phase shift between the sub-pupils. First, I improved the control software of the testbed to increase its speed and to allow the acquisition of interferometric data requiring the synchronised control of the different components for the modulation of the fringes (deformable mirror, delay lines and camera).

Then I characterised the optical properties of two photonic chips that use two different interferometric beam recombination techniques, such as their transmission, the cross-talk between the waveguides, their behaviour in the two polarisations and the instrumental contrast. I built an optical system simulating a binary source on the testbed, which allows to inject both a source with a wide spectral band to simulate a central star and a source with a narrow spectral band to simulate an exoplanet with an emission line. This was crucial to demonstrate the instrument's ability to detect and measure the companion signal. To measure the latter, I took advantage of the properties of spectral differential phase, which is a self-calibrated quantity. Indeed, the measurement of the phase over a wide spectral band allows the calibration of the phase measurements by the continuum signal highlighting a possible signal in the emission line of interest. Thus, I developed a data processing and analysis program allowing to estimate the differential phase as well as the complex visibility and the closure phase and to fit them with a star-companion model. Therefore I was able to demonstrate that FIRSTv2 could detect a protoplanet-like companion at an equivalent separation of $0.7 \lambda / B$ from the central source, with a contrast of up to $0.1$.

One of the photonic chips was integrated to the FIRST instrument installed on the SCExAO bench, resulting in the first light from FIRSTv2 on 10 September 2021. I acquired on-sky data during several remote observation nights as well as during a mission to Hawaii in February 2022. It was the opportunity for me to integrate the second component for comparison with the first, as well as to deploy the control software developed in the laboratory. The results of the processing of these on-sky data allowed me to learn more about the acquisition and processing methods of interferometric data.

To conclude, I characterised and studied the performance of the integrated optics technology in the context of high contrast and high resolution interferometric imaging in the visible range. I developed the FIRSTv2 control software in the laboratory before deploying it on the SCExAO bench for the first light. Then, I developed the interferometric data processing and analysis program for FIRSTv2, including a fitting of the interferometric observables by a protoplanet model with a strong emission line in its spectrum. The objective is to characterise the accretion mechanisms of young exoplanetary systems in formation. Finally, I have had the opportunity to participate in many observing nights during which I have acquired data on simple targets such as companion-star binaries with low contrast.

