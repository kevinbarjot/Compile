%----------------------------------------------------------------------------------------
%	FRENCH
%----------------------------------------------------------------------------------------

\newpage
\thispagestyle{empty}
\chapter*{Résumé court}

\noindent\rule[2pt]{\textwidth}{0.5pt}
\begin{center}
    \large\textbf{\ttitle\\}
\end{center}
    \footnotesize J’ai travaillé pendant ma thèse dans le domaine de l’imagerie à haut contraste et à haute résolution angulaire (HRA) pour l’étude des protoplanètes. L'observation des protoplanètes est crucial pour l'étude des mécanismes de la formation planétaire. Ces jeunes planètes se caractérisent par la présence de lignes d'émission dans leur spectre. Celle qui nous intéresse est la raie Ha et le contraste du système est alors plus faible à ces longueurs d’onde et donc plus accessible. L’instrument Fibered Interferometer foR a Single Telescope (FIRST) utilise le concept de réarrangement de pupille fibré dans le visible, intégré sur le banc d’optique adaptative extrême Subaru Coronagraphic Extreme Adaptive Optics (SCExAO) du télescope Subaru. Son principe est de sous-divisé la pupille d’entrée en sous-pupilles dont les faisceaux sont injectés dans les fibres optiques monomodes. Celles-ci ont le double intérêt d’appliquer un filtrage spatial du front d’onde supprimant ainsi les aberrations optiques à l’échelle des sous-pupilles, en même temps que le réarrangement des sous-pupilles. Les interférogrammes résultant de l’interférence des faisceaux de ces dernières sont ensuite mesurés. FIRST a démontré qu’il permettait l’imagerie de binaires stellaires à une résolution en-deçà de la limite de diffraction du télescope avec ce concept. La réplique d’une nouvelle version de cet instrument (FIRSTv2) a été construite en laboratoire afin de permettre les développements de la technologie d’optique intégrée. Mon travail de thèse est d’évaluer ses performances et sa faisabilité pour l’imagerie HRA, dans le but d’améliorer les performances en contraste de l’instrument. La recombinaison de chaque paire de sous-pupilles est ainsi codée sur une sortie de la puce photonique avant d’être dispersée et imagée sur quelques pixels de la caméra. L’échantillonnage des franges se fait par modulation temporelle des chemins optiques des faisceaux par les segments du miroir déformable. De plus, j’ai continué le développement du banc de test, aussi bien au niveau du montage optique que du logiciel de contrôle, de mettre au point une procédure spécifique d’acquisition des données et de développer un programme de traitement et d’analyse de données. Cette analyse des données s’effectue via le calcul de la phase différentielle spectrale, qui est une observable auto-étalonnée des perturbations atmosphériques et instrumentales. Pour cela, les mesures de phase des visibilités complexes de chaque base sont étalonnées par le signal du continuum mettant en évidence un possible signal dans la raie d’intérêt. Ainsi, à partir d’un simulateur de source protoplanétaire que j’ai intégré sur le banc, j’ai pu démontrer que FIRSTv2 pouvait détecter un compagnon de type protoplanètaire à une séparation équivalente à 0.7 l / B de la source centrale, avec un contraste d’environ 0.5. Enfin, j’ai participé à l’intégration et à la première lumière de FIRSTv2 sur le banc SCExAO pour laquelle le logiciel développé en laboratoire a été déployé. Les tests et les données acquises lors de quelques nuits d’observations montrent qu’une isolation des fibres optiques sur le banc est nécessaire et que la méthode d’acquisition des interférogrammes par modulation temporelle doit probablement être changée pour une modulation dans la puce en changeant la technologie de recombinaison pour une ABCD.

\vspace{0.2cm}
\noindent{\normalsize\textbf{Mots clés :}} \keywordnamesfr

\noindent\rule[2pt]{\textwidth}{0.5pt}


%----------------------------------------------------------------------------------------
%	ENGLISH
%----------------------------------------------------------------------------------------


\chapter*{Short abstract}

\noindent\rule[2pt]{\textwidth}{0.5pt}
\begin{center}
    \large\textbf{Characterization and deployement of Photonic Integrated Circuits for the FIRST fibered interferometer at the Subaru Telescope in the context of accreting protoplanets studies\\}
\end{center}
    \footnotesize During my thesis, I worked in the field of high contrast and high angular resolution (HRA) imaging for the study of protoplanets. The observation of protoplanets is key to constrain the mechanisms of planetary formation. These young planets are characterized by the presence of emission lines in their spectrum. The one we are interested in is the Ha line and the contrast of the system is then weaker at these wavelengths and thus more accessible. The Fibered Interferometer foR a Single Telescope (FIRST) instrument uses the concept of fibered pupil remapping in the visible, integrated on the Subaru Coronagraphic Extreme Adaptive Optics (SCExAO) telescope. Its principle is to sub-divide the entrance pupil into sub-pupils whose beams are injected into single mode optical fibers. The latest have the double advantage of applying a spatial filtering on the wavefront thus suppressing the optical aberrations at the scale of the sub-pupils, at the same time as the remapping of the sub-pupils. The interferograms resulting from the interference of the sub-pupil beams are then measured. FIRST has demonstrated that it can detect stellar binaries at a resolution below the diffraction limit of the telescope with this concept. The replica of a new version of this instrument (FIRSTv2) has been built in laboratory to allow the development of the integrated optics technology. My thesis work is to evaluate its performance and its feasibility for HRA imaging, in order to improve the contrast performance of the instrument. The recombination of each pair of sub-pupils is encoded on an output of the photonic chip then dispersed and imaged on few pixels of the camera. The fringe sampling is performed by temporal modulation of the sub-pupil optical paths by the segments of the deformable mirror. In addition, I continued the development of the testbed, both in terms of the optical assembly and the control software, to develop a specific procedure for data acquisition and to develop a program for data reduction and analysis. The goal of this data analysis  is to calculate the spectral differential phase, which is a self-calibrated observable of the atmospheric and instrumental perturbations. In that purpose, the phase measurements of the complex visibilities of each baseline are calibrated by the continuum signal highlighting a possible signal in the line of interest. Thus, from a protoplanetary source simulator that I integrated on the testbed, I was able to demonstrate that FIRSTv2 could detect a protoplanetary companion at a separation equivalent to 0.7 l / B from the central source, with a contrast of about 0.5. Finally, I participated in the integration and the first light of FIRSTv2 on the SCExAO bench for which the software developed in laboratory was deployed. The tests and data acquired during several observation nights show that an isolation of the optical fibers on the bench is necessary and that the method of acquisition of the interferograms by temporal modulation may need to be changed for an on-chip modulation by changing the recombination technology using an ABCD chip.

\vspace{0.2cm}
\noindent{\normalsize\textbf{Keywords\string:}} \keywordnamesen

\noindent\rule[2pt]{\textwidth}{0.5pt}


