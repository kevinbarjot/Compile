%----------------------------------------------------------------------------------------
%	Language and font encodings
%----------------------------------------------------------------------------------------


\usepackage[utf8x]{inputenc}
\usepackage[T1]{fontenc}
\usepackage[round]{natbib}
\usepackage[french]{babel}
% french: remove space character after comas in equations
% \DecimalMathComma               


%----------------------------------------------------------------------------------------
%	Sets page size and margins
%----------------------------------------------------------------------------------------


\usepackage{geometry}

%% Pour PDF
\geometry{
	paper=a4paper, % Change to letterpaper for US letter
	% 2,5 cm et 3 cm => textwidth = 21 - 5,5 = 15,5 cm
	inner=2.5cm, % Inner margin
	outer=2.5cm, % Outer margin
	bindingoffset=.5cm, % Binding offset
	top=1.5cm, % Top margin
	bottom=2cm, % Bottom margin
	%showframe, % Uncomment to show how the type block is set on the page
	headheight=4ex,
	includehead,
	includefoot
}

%% Pour impression : avec 2 mm de "fond perdu" à gauche et à droite (version pour rapporteurs avec découpe manuelle)
%\geometry{
%	papersize={21.4cm,29.7cm},
%	% 2,5 cm et 3 cm => textwidth = 21 - 5,5 = 15,5 cm
%	inner=2.7cm, % Inner margin
%	outer=2.7cm, % Outer margin
%	bindingoffset=.5cm, % Binding offset
%	top=1.5cm, % Top margin
%	bottom=2cm, % Bottom margin
%	%showframe, % Uncomment to show how the type block is set on the page
%	headheight=4ex,
%	includehead,
%	includefoot
%}

%% Pour impression : avec 5 mm de "fond perdu" sur chaque côté (version pour repro)
%\geometry{
% 	papersize={22.0cm,30.7cm},
% 	% 2,5 cm et 3 cm => textwidth = 21 - 5,5 = 15,5 cm
% 	inner=3.0cm, % Inner margin
% 	outer=3.0cm, % Outer margin
% 	bindingoffset=.5cm, % Binding offset
% 	top=2.0cm, % Top margin
% 	bottom=2.5cm, % Bottom margin
% 	%showframe, % Uncomment to show how the type block is set on the page
% 	headheight=4ex,
% 	includehead,
% 	includefoot
%}

\setlength{\parindent}{15pt}


%----------------------------------------------------------------------------------------
%	Useful packages
%----------------------------------------------------------------------------------------


\usepackage{calc}
\usepackage{ulem}  % to underline text
\usepackage{pdfpages}  % to integrate pdf pages in the document
\usepackage{hyphenat}  % to specify when hyphenating ('\hyph{}')
\usepackage[autostyle=true]{csquotes}
\usepackage{rotating}


%----------------------------------------------------------------------------------------
%	Maths and symbols
%----------------------------------------------------------------------------------------


\usepackage{icomma}
\usepackage{amsmath}
\usepackage{amssymb}
\usepackage{stmaryrd}  % for double brackets (\llbracket, \rrbracket...)
\setcounter{MaxMatrixCols}{20}
\usepackage{upgreek}  % to display greek letters not in italic in math expressions
\usepackage{marvosym}  % to get the euro symbols
\usepackage{tikz}  % To draw arrows on the side of matrices
\newcommand{\tikzmark}[1]{\tikz[overlay, remember picture] \coordinate (#1);}


%----------------------------------------------------------------------------------------
%	Figures
%----------------------------------------------------------------------------------------


\usepackage{graphicx}
\usepackage[format=plain,labelfont={bf},labelsep=endash]{caption}
\usepackage{subcaption}
% Width of all figure (not subfigures)
\newcommand{\figwidth}{\textwidth}
\usepackage[pagebackref]{hyperref}


%----------------------------------------------------------------------------------------
%	Tables
%----------------------------------------------------------------------------------------


\usepackage{multirow}  % for tables
\renewcommand{\arraystretch}{1.2}


%----------------------------------------------------------------------------------------
%	Mini toc
%----------------------------------------------------------------------------------------


\usepackage[nottoc]{tocbibind}
\usepackage{minitoc}
\setcounter{minitocdepth}{3}

% Space after minitoc
\newlength{\spaceafterminitoc}
\setlength{\spaceafterminitoc}{0.7cm}

% Change the title of minitoc
\addto{\captionsfrench}{% Making babel aware of special titles
  \renewcommand{\mtctitle}{Sommaire}
}


%----------------------------------------------------------------------------------------
%	Acronym settings
%----------------------------------------------------------------------------------------


\usepackage[acronym,toc,shortcuts]{glossaries}
\renewcommand*{\glstextformat}[1]{\textcolor{black}{#1}} % to set the link color of acronyms in text
\makeglossaries
%----------------------------------------------------------------------------------------
%	List of acronym definitions
%----------------------------------------------------------------------------------------


\newacronym{IAU}{IAU}{International Astronomical Union}
\newacronym{FIRST}{FIRST}{Fibered Interferometer foR a Single Telescope}
\newacronym{FIRSTv1}{FIRSTv1}{Fibered Interferometer foR a Single Telescope version~1}
\newacronym{FIRSTv2}{FIRSTv2}{Fibered Interferometer foR a Single Telescope version~2}
\newacronym{LESIA}{LESIA}{Laboratoire d'\'Etudes Spatiales et d'Instrumentation en Astrophysique}
\newacronym{HRA}{HRA}{haut contraste et Haute R\'esolution Angulaire}
\newacronym{SCExAO}{SCExAO}{Subaru Coronagraphic Extreme Adaptive Optics}
\newacronym{CACAO}{CACAO}{Control And Compute for Adaptive Optics}
\newacronym{V2PM}{V2PM}{Visibility To Pixel Matrix}
\newacronym{P2VM}{P2VM}{Pixel To Visibility Matrix}
\newacronym{SVD}{SVD}{Singular Value Decomposition}
\newacronym{CMOS}{CMOS}{Complementary Metal Oxide Semiconductor}
\newacronym{ADU}{ADU}{Analog-Digital Unit}
\newacronym{OPD}{OPD}{Optical Path Difference}
\newacronym{ODL}{ODL}{Optical Delay Line}
\newacronym{MEMS}{MEMS}{Micro-ElectroMechanical Systems}
\newacronym{PSF}{PSF}{Point Spread Function}
\newacronym{UV}{UV}{UltraViolet}
\newacronym{CA}{CA}{Core Accretion}
\newacronym{GI}{GI}{Gravitational Instabilities}
\newacronym{TTS}{TTS}{T Tauri Star}
\newacronym{CTTS}{CTTS}{Classical T Tauri Star}
\newacronym{WTTS}{WTTS}{Weak T Tauri Star}
\newacronym{MUSE}{MUSE}{Multi Unit Spectroscopic Explorer}
\newacronym{VLT}{VLT}{Very Large Telescope}
\newacronym{VLTI}{VLTI}{Very Large Telescope Interferometer}
\newacronym{ELT}{ELT}{Extremely Large Telescope}
\newacronym{NIRC2}{NIRC2}{Near InfraRed Camera 2}
\newacronym{MagAO}{MagAO}{Magellan Adaptive Optics}
\newacronym{MagAO-X}{MagAO-X}{Magellan Adaptive Optics - eXtreme}
\newacronym{LMIRCam}{LMIRCam}{Large binocular telescope Mid-InfraRed Camera}
\newacronym{LBTI}{LBTI}{Large Binocular Telescope Interferometer}
\newacronym{ISIS}{ISIS}{Intermediate-dispersion Spectrograph and Imaging System}
\newacronym{WHT}{WHT}{William Herschel Telescope}
\newacronym{CHARIS}{CHARIS}{Coronagraphic High Angular Resolution Imaging Spectrograph}
\newacronym{SPHERE}{SPHERE}{Spectro-Polarimetic High contrast imager for Exoplanets REsearch}
\newacronym{NACO}{NACO}{Nasmyth Adaptive Optics System - Coude Near Infrared Camera}
\newacronym{GPI}{GPI}{Gemini Planet Imager}
\newacronym{GST}{GST}{Gemini South Telescope}
\newacronym{GAPlanetS}{GAPlanetS}{Giant Accreting Protoplanet Survey}
\newacronym{AMBER}{AMBER}{Astronomical Multi-BEam combineR}
\newacronym{GRAVITY}{GRAVITY}{General Relativity Analysis via VLT InTerferometrY}
\newacronym{SNR}{SNR}{Signal-to-Noise Ratio}
\newacronym{BCD}{BCD}{Beam Commutation Device}
\newacronym{JWST}{JWST}{James Webb Space Telescope}
\newacronym{MIRI}{MIRI}{Mid-Infrared Instrument}
\newacronym{MFD}{MFD}{Mode Field Diameter}
\newacronym{VPH}{VPH}{Volume Phase Holographic}
\newacronym{pyMILK}{pyMILK}{PYthon Multi-purpose Imaging Libraries toolKit}
\newacronym{MILK}{MILK}{Multi-purpose Imaging Libraries toolKit}
\newacronym{SDK}{SDK}{Software Development Kit}
\newacronym{SHM}{SHM}{SHared Memory}
\newacronym{IPAG}{IPAG}{Institut de Plan\'etologie et d'Astrophysique de Grenoble}
\newacronym{RAM}{RAM}{Random Access Memory}
\newacronym{CCD}{CCD}{Charge Coupled Device}
\newacronym{EMCCD}{EMCCD}{Electron Multiplying Charge Coupled Device}
\newacronym{GLINT}{GLINT}{Guided-Light Interferometric Nulling Technology}
\newacronym{AU}{AU}{Astronomical Unit}
\newacronym{UA}{UA}{Unit\'e Astronomique}
\newacronym{TESS}{TESS}{Transiting Exoplanet Survey Satellite}
\newacronym{ADI}{ADI}{Angular Differential Imaging}
\newacronym{SDI}{SDI}{Spectral Differential Imaging}
\newacronym{RDI}{RDI}{Reference Star Differential Imaging}
\newacronym{TMT}{TMT}{Thirty Meter Telescope}
\newacronym{LBTAO}{LBTAO}{Large Binocular Telescope Adaptive Optics}
\newacronym{CHARA}{CHARA}{Center for High Angular Resolution Astronomy}
\newacronym{SAM}{SAM}{Sparse Aperture Masking}
\newacronym{NIRISS}{NIRISS}{Near InfraRed Imager and Slitless Spectrograph}
\newacronym{MICADO}{MICADO}{Multi-ao Imaging CAmera for Deep Observations}
\newacronym{HST}{HST}{Hubble Space Telescope}
\newacronym{NICI}{NICI}{Near-Infrared Coronagraphic Imager}
\newacronym{VAMPIRES}{VAMPIRES}{Visible Aperture Masking Polarimetric Imager for Resolved Exoplanetary Structures}
\newacronym{REACH}{REACH}{Rigorous Exoplanetary Atmosphere Characterization}
\newacronym{MEC}{MEC}{MKID Exoplanet Camera}
\newacronym{PDI}{PDI}{Polarization Differential Imaging}
\newacronym{RHEA}{RHEA}{Replicable High resolution Exoplanet and Asteroseismology}
\newacronym{PIAA}{PIAA}{Phase-Induced Amplitude Apodization}
\newacronym{IR}{IR}{InfraRouge}
\newacronym{WDM}{WDM}{Wavelength Division Multiplexer}
\newacronym{ExAO}{ExAO}{Extreme Adaptive Optics}


% \newcommand{\aclegend}[1]{(\acs{#1} - \textit{\acl{#1}})}

% https://www.overleaf.com/learn/latex/Glossaries#Compiling_the_glossary
% If you are compiling the document (called main.tex) using pdflatex on your local machine, you have to use these commands:
% pdflatex main.tex
% makeglossaries main
% pdflatex main.tex


%----------------------------------------------------------------------------------------
%	Figure, table and equation numbering
%----------------------------------------------------------------------------------------


\renewcommand{\thefigure}{\thechapter.\arabic{figure}}
\renewcommand{\thetable}{\thechapter.\arabic{table}}
\renewcommand{\theequation}{\thechapter.\arabic{equation}}


%----------------------------------------------------------------------------------------
%	Create a third level of subsection
%----------------------------------------------------------------------------------------


\usepackage{titlesec}
\setcounter{secnumdepth}{4}
\titleformat{\paragraph}{\normalfont\normalsize\bfseries}{\theparagraph}{1em}{}
\titlespacing*{\paragraph}{0pt}{3.25ex plus 1ex minus .2ex}{1.5ex plus .2ex}
\newcommand{\threesubsection}{\paragraph}
\newcommand{\mysubparagraph}[1]{\subparagraph{#1}\mbox{}\\ \mbox{}}


%----------------------------------------------------------------------------------------
%	Abbreviations
%----------------------------------------------------------------------------------------


\newcommand{\vv}[1]{\overrightarrow{#1}}
\newcommand{\ha}[0]{\text{$\text{H}\upalpha$}}
\newcommand{\Lha}[0]{\text{$\text{L}_{\ha}$}}
\newcommand{\Lacc}[0]{\text{$\text{L}_{\text{acc}}$}}
\newcommand{\Wd}[0]{\text{$\text{W}_{10}$}}
\newcommand{\sk}[0]{\textit{Super K}}
\newcommand{\wiggles}[0]{\textit{wiggles}}
\newcommand{\degree}[0]{^{\circ}}
\newcommand{\um}{$\upmu$m}
\newcommand{\Like}[0]{\mathcal{L}}
\newcommand{\MJ}[0]{\text{$\text{M}_\text{J}$}}
\newcommand{\MS}[0]{\text{$\text{M}_{\odot}$}}


%----------------------------------------------------------------------------------------
%	Colored comments
%----------------------------------------------------------------------------------------


\usepackage{xcolor}
\newcommand{\comments}[1]{\textcolor{orange}{#1}}
\newcommand{\kevinco}[1]{\textcolor{red}{#1}}
\newcommand{\elsaco}[1]{\textcolor{blue}{#1}}
\newcommand{\sylvestreco}[1]{\textcolor{green}{#1}}
\usepackage{comment} % to comment parts of the text with \begin{comment} \end{comment}


%----------------------------------------------------------------------------------------
%	
%----------------------------------------------------------------------------------------

