%----------------------------------------------------------------------------------------
%	Language and font encodings
%----------------------------------------------------------------------------------------


\usepackage[round]{natbib}
\usepackage[utf8x]{inputenc}
\usepackage[T1]{fontenc}
\usepackage[french]{babel}


%----------------------------------------------------------------------------------------
%	Sets page size and margins
%----------------------------------------------------------------------------------------


% \usepackage[a4paper,top=1.5cm,bottom=2cm,left=2.5cm,right=2.5cm,marginparwidth=1.75cm]{geometry}
\usepackage[inner=3.5cm,
	    outer=2.5cm,
	    tmargin=2cm, 
	    bmargin=2cm, 
	    includefoot,
	    includehead,
	    headheight=28pt]{geometry}


%----------------------------------------------------------------------------------------
%	Useful packages
%----------------------------------------------------------------------------------------


\usepackage{amsmath}
\usepackage{amssymb}
\setcounter{MaxMatrixCols}{20}
\usepackage{upgreek}  % to display greek letters not in italic in math expressions
\usepackage{graphicx}
\usepackage[hang,labelfont=it]{caption}
\usepackage{subcaption}
\usepackage[colorlinks=true, allcolors=blue]{hyperref}  % to reference sections, figures, etc...
\usepackage{ulem}  % to underline text
\usepackage{multirow}  % for tables
\renewcommand{\arraystretch}{1.3}
\usepackage{pdfpages}  % to integrate pdf pages in the document
\usepackage{marvosym}  % to get the euro symbols
\usepackage{hyphenat}  % to specify when hyphenating ('\hyph{}')


%----------------------------------------------------------------------------------------
%	Fancy headers
%----------------------------------------------------------------------------------------


% %----------------------------------------------------------------------------------------
%	Fancy headers
%----------------------------------------------------------------------------------------


\usepackage{fancyhdr}
\pagestyle{fancy}
\fancyhf{}
\fancyhead[LE,RO]{\bfseries \thepage}
\fancyhead[LO]{\nouppercase \rightmark}
\fancyhead[RE]{\nouppercase \leftmark}
\renewcommand{\headrulewidth}{1pt}
% \renewcommand{\headheight}{20pt}
% \addtolength{\headheight}{10pt} % space for the rule

%redéfini le "plain" car c'est ce qui est utilisé automatiquement pour la première page d'un chapitre
\fancypagestyle{plain}{
    \fancyhf{}
    \fancyhead[LE,RO]{\bfseries \thepage}
    \renewcommand{\headrulewidth}{1pt}
%     \addtolength{\headheight}{20pt} % space for the rule
    }



%----------------------------------------------------------------------------------------
%	Acronym settings
%----------------------------------------------------------------------------------------


\usepackage[acronym,toc,shortcuts]{glossaries}
\renewcommand*{\glstextformat}[1]{\textcolor{black}{#1}} % to set the link color of acronyms in text
\makeglossaries
%----------------------------------------------------------------------------------------
%	List of acronym definitions
%----------------------------------------------------------------------------------------


\newacronym{FIRST}{FIRST}{Fibered Interferometer foR a Single Telescope}
\newacronym{FIRSTv1}{FIRSTv1}{Fibered Interferometer foR a Single Telescope version 1}
\newacronym{FIRSTv2}{FIRSTv2}{Fibered Interferometer foR a Single Telescope version 2}
\newacronym{LESIA}{LESIA}{Laboratoire d'\'Etudes Spatiales et d'Instrumentation en Astrophysique}
\newacronym{SCExAO}{SCExAO}{Subaru Coronagraphic Extreme Adaptive Optics}
\newacronym{V2PM}{V2PM}{Visibility To Pixel Matrix}
\newacronym{P2VM}{P2VM}{Pixel To Visibility Matrix}
\newacronym{SVD}{SVD}{Singular Value Decomposition}
\newacronym{CMOS}{CMOS}{Complementary Metal Oxide Semiconductor}
\newacronym{ADU}{ADU}{Analog-Digital Unit}
\newacronym{OPD}{OPD}{Optical Path Difference}
\newacronym{ODL}{ODL}{Optical Delay Line}
\newacronym{MEMS}{MEMS}{Micro-ElectroMechanical Systems}
\newacronym{PSF}{PSF}{Point Spread Function}
\newacronym{UV}{UV}{UltraViolet}
\newacronym{CA}{CA}{Core Accretion}
\newacronym{GI}{GI}{Gravitational Instabilities}
\newacronym{TTS}{TTS}{T Tauri Star}
\newacronym{CTTS}{CTTS}{Classical T Tauri Star}
\newacronym{WTTS}{WTTS}{Weak T Tauri Star}
\newacronym{MUSE}{MUSE}{Multi Unit Spectroscopic Explorer}
\newacronym{VLT}{VLT}{Very Large Telescope}
\newacronym{ELT}{ELT}{Extremely Large Telescope}
\newacronym{NIRC2}{NIRC2}{Near InfraRed Camera 2}
\newacronym{MagAO}{MagAO}{Magellan Adaptive Optics}
\newacronym{LMIRCam}{LMIRCam}{Large binocular telescope Mid-InfraRed Camera}
\newacronym{LBTI}{LBTI}{Large Binocular Telescope Interferometer}
\newacronym{ISIS}{ISIS}{Intermediate-dispersion Spectrograph and Imaging System}
\newacronym{WHT}{WHT}{William Herschel Telescope}
\newacronym{CHARIS}{CHARIS}{Coronagraphic High Angular Resolution Imaging Spectrograph}
\newacronym{SPHERE}{SPHERE}{Spectro-Polarimetic High contrast imager for Exoplanets REsearch}
\newacronym{NACO}{NACO}{Nasmyth Adaptive Optics System - Coude Near Infrared Camera}
\newacronym{GPI}{GPI}{Gemini Planet Imager}
\newacronym{GST}{GST}{Gemini South Telescope}
\newacronym{GAPlanetS}{GAPlanetS}{Giant Accreting Protoplanet Survey}
\newacronym{AMBER}{AMBER}{Astronomical Multi-BEam combineR}
\newacronym{GRAVITY}{GRAVITY}{General Relativity Analysis via VLT InTerferometrY}
\newacronym{SNR}{SNR}{Signal-to-Noise Ratio}
\newacronym{BCD}{BCD}{Beam Commutation Device}
\newacronym{JWST}{JWST}{James Webb Space Telescope}
\newacronym{MIRI}{MIRI}{Mid-Infrared Instrument}
\newacronym{MFD}{MFD}{Mode Field Diameter}
\newacronym{VPH}{VPH}{Volume Phase Holographic}
\newacronym{pyMILK}{pyMILK}{PYthon Multi-purpose Imaging Libraries toolKit}
\newacronym{MILK}{MILK}{Multi-purpose Imaging Libraries toolKit}
\newacronym{SDK}{SDK}{Software Development Kit}
\newacronym{SHM}{SHM}{SHared Memory}
\newacronym{IPAG}{IPAG}{Institut de Plan\'etologie et d'Astrophysique de Grenoble}
\newacronym{RAM}{RAM}{Random Access Memory}
\newacronym{CCD}{CCD}{Charge Coupled Device}
\newacronym{GLINT}{GLINT}{Guided-Light Interferometric Nulling Technology}
% \newacronym{}{}{}

% \newcommand{\aclegend}[1]{(\acs{#1} - \textit{\acl{#1}})}

% https://www.overleaf.com/learn/latex/Glossaries#Compiling_the_glossary
% If you are compiling the document (called main.tex) using pdflatex on your local machine, you have to use these commands:
% pdflatex main.tex
% makeglossaries main
% pdflatex main.tex


%----------------------------------------------------------------------------------------
%	Figure, table and equation numbering
%----------------------------------------------------------------------------------------


\renewcommand{\thefigure}{\thesection.\arabic{figure}}
\renewcommand{\thetable}{\thesection.\arabic{table}}
\renewcommand{\theequation}{\thesection.\arabic{equation}}


%----------------------------------------------------------------------------------------
%	Create a third level of subsection
%----------------------------------------------------------------------------------------


\usepackage{titlesec}
\setcounter{secnumdepth}{4}
\titleformat{\paragraph}{\normalfont\normalsize\bfseries}{\theparagraph}{1em}{}
\titlespacing*{\paragraph}{0pt}{3.25ex plus 1ex minus .2ex}{1.5ex plus .2ex}
\newcommand{\threesubsection}{\paragraph}
\newcommand{\mysubparagraph}[1]{\subparagraph{#1}\mbox{}\\}


%----------------------------------------------------------------------------------------
%	Abbreviations
%----------------------------------------------------------------------------------------


\newcommand{\vv}[1]{\overrightarrow{#1}}
\newcommand{\ha}[0]{H$\alpha$}
\newcommand{\sk}[0]{\textit{Super K}}
\newcommand{\wiggles}[0]{\textit{wiggles}}
\newcommand{\degree}[0]{^{\circ}}
\newcommand{\um}{$\upmu$m}
\newcommand{\Like}{\mathcal{L}}


%----------------------------------------------------------------------------------------
%	Colored comments
%----------------------------------------------------------------------------------------


\usepackage{xcolor}
\newcommand{\comments}[1]{\textcolor{orange}{#1}}
\newcommand{\kevinco}[1]{\textcolor{red}{#1}}
\newcommand{\elsaco}[1]{\textcolor{blue}{#1}}
\newcommand{\sylvestreco}[1]{\textcolor{green}{#1}}
\usepackage{comment} % to comment parts of the text with \begin{comment} \end{comment}


%----------------------------------------------------------------------------------------
%	
%----------------------------------------------------------------------------------------

