%----------------------------------------------------------------------------------------
%	FIRST PARGE ENVIRONMENT
%----------------------------------------------------------------------------------------


\newenvironment{mytitlepage}
{
}%
{
}

% Logos
\newcommand{\logoupc}{
	\hspace{-2cm}\raisebox{-2cm}{\includegraphics[width=6cm]{Openings/logo_upc.png}}
}


%----------------------------------------------------------------------------------------
%	BOXES FOR CHAPTERS
%----------------------------------------------------------------------------------------


\newlength{\ongletwidth}
\setlength{\ongletwidth}{3cm}% 1.5cm apparent sur PDF A4 (coupé)

\newlength{\ongletheight}

\newcommand{\ongletfontI}{%
	\fontfamily{pag}\fontseries{b}\fontshape{n}\fontsize{1cm}{1cm}\selectfont\color{white}
}

\newcommand{\ongletfontII}{%
	\fontfamily{pag}\fontseries{b}\fontshape{n}\fontsize{1cm}{1cm}\selectfont\color{white}
} 

\newcommand{\ongletfont}{%	
	\ifnum\pdfstrcmp{\thechapter}{A}=0 % teste si les 2 strings sont indentiques
		\ongletfontII%
	\else\ifnum\pdfstrcmp{\thechapter}{B}=0 % teste si les 2 strings sont indentiques
		\ongletfontII% 
	\else\ifnum\pdfstrcmp{\thechapter}{C}=0 % teste si les 2 strings sont indentiques
		\ongletfontII% 
	\else\ifnum\pdfstrcmp{\thechapter}{D}=0 % teste si les 2 strings sont indentiques
		\ongletfontII% 
	\else\ifnum\pdfstrcmp{\thechapter}{E}=0 % teste si les 2 strings sont indentiques
		\ongletfontII%
	\else\ifnum\pdfstrcmp{\thechapter}{F}=0 % teste si les 2 strings sont indentiques
		\ongletfontII%
	\else\ifnum\pdfstrcmp{\thechapter}{G}=0 % teste si les 2 strings sont indentiques
		\ongletfontII%
	\else\ifnum\pdfstrcmp{\thechapter}{H}=0 % teste si les 2 strings sont indentiques
		\ongletfontII%
	\else\ifnum\pdfstrcmp{\thechapter}{I}=0 % teste si les 2 strings sont indentiques
		\ongletfontII%
	\else\ifnum\pdfstrcmp{\thechapter}{J}=0 % teste si les 2 strings sont indentiques
		\ongletfontII%
	\else\ifnum\pdfstrcmp{\thechapter}{K}=0 % teste si les 2 strings sont indentiques
		\ongletfontII%
	\else\ifnum\pdfstrcmp{\thechapter}{L}=0 % teste si les 2 strings sont indentiques
		\ongletfontII%
	\else\ifnum\pdfstrcmp{\thechapter}{M}=0 % teste si les 2 strings sont indentiques
		\ongletfontII%
	\else\ifnum\pdfstrcmp{\thechapter}{N}=0 % teste si les 2 strings sont indentiques
		\ongletfontII%
	\else
		\ongletfontI%
	\fi\fi\fi\fi\fi\fi\fi\fi\fi\fi\fi\fi\fi\fi
}

% To modify: set the number of chapters before the conclusion to shift appendices onglet
\newcommand{\conclusionchapternumber}{6}

\newcommand{\valueforonglet}{%	
	%\ifnum\value{chapter}=0
	\ifnum\pdfstrcmp{\thechapter}{A}=0 % teste si les 2 strings sont indentiques
		\conclusionchapternumber%
	\else\ifnum\pdfstrcmp{\thechapter}{B}=0 % teste si les 2 strings sont indentiques
		\conclusionchapternumber%
	\else\ifnum\pdfstrcmp{\thechapter}{C}=0 % teste si les 2 strings sont indentiques
		\conclusionchapternumber% 
	\else\ifnum\pdfstrcmp{\thechapter}{D}=0 % teste si les 2 strings sont indentiques
		\conclusionchapternumber% 
	\else\ifnum\pdfstrcmp{\thechapter}{E}=0 % teste si les 2 strings sont indentiques
		\conclusionchapternumber%
	\else\ifnum\pdfstrcmp{\thechapter}{F}=0 % teste si les 2 strings sont indentiques
		\conclusionchapternumber%
	\else\ifnum\pdfstrcmp{\thechapter}{G}=0 % teste si les 2 strings sont indentiques
		\conclusionchapternumber%
	\else\ifnum\pdfstrcmp{\thechapter}{H}=0 % teste si les 2 strings sont indentiques
		\conclusionchapternumber%
	\else\ifnum\pdfstrcmp{\thechapter}{I}=0 % teste si les 2 strings sont indentiques
		\conclusionchapternumber%
	\else\ifnum\pdfstrcmp{\thechapter}{J}=0 % teste si les 2 strings sont indentiques
		\conclusionchapternumber%
	\else\ifnum\pdfstrcmp{\thechapter}{K}=0 % teste si les 2 strings sont indentiques
		\conclusionchapternumber%
	\else\ifnum\pdfstrcmp{\thechapter}{L}=0 % teste si les 2 strings sont indentiques
		\conclusionchapternumber%
	\else\ifnum\pdfstrcmp{\thechapter}{M}=0 % teste si les 2 strings sont indentiques
		\conclusionchapternumber%
	\else\ifnum\pdfstrcmp{\thechapter}{N}=0 % teste si les 2 strings sont indentiques
		\conclusionchapternumber%
	\else
		\thechapter%
	\fi\fi\fi\fi\fi\fi\fi\fi\fi\fi\fi\fi\fi\fi
}

\newcommand{\coulorforonglet}{%	
	%\ifnum\value{chapter}=0
	\ifnum\pdfstrcmp{\thechapter}{A}=0% teste si les 2 strings sont indentiques
		colorboxchap%
	\else\ifnum\pdfstrcmp{\thechapter}{B}=0% teste si les 2 strings sont indentiques
		colorboxchap% 
	\else\ifnum\pdfstrcmp{\thechapter}{C}=0% teste si les 2 strings sont indentiques
		colorboxchap% 
	\else\ifnum\pdfstrcmp{\thechapter}{D}=0% teste si les 2 strings sont indentiques
		colorboxchap%
	\else\ifnum\pdfstrcmp{\thechapter}{E}=0% teste si les 2 strings sont indentiques
		colorboxchap%
	\else\ifnum\pdfstrcmp{\thechapter}{F}=0% teste si les 2 strings sont indentiques
		colorboxchap%
	\else\ifnum\pdfstrcmp{\thechapter}{G}=0% teste si les 2 strings sont indentiques
		colorboxchap%
	\else\ifnum\pdfstrcmp{\thechapter}{H}=0% teste si les 2 strings sont indentiques
		colorboxchap%
	\else\ifnum\pdfstrcmp{\thechapter}{I}=0% teste si les 2 strings sont indentiques
		colorboxchap%
	\else\ifnum\pdfstrcmp{\thechapter}{J}=0% teste si les 2 strings sont indentiques
		colorboxchap%
	\else\ifnum\pdfstrcmp{\thechapter}{K}=0% teste si les 2 strings sont indentiques
		colorboxchap%
	\else\ifnum\pdfstrcmp{\thechapter}{L}=0% teste si les 2 strings sont indentiques
		colorboxchap%
	\else\ifnum\pdfstrcmp{\thechapter}{M}=0% teste si les 2 strings sont indentiques
		colorboxchap%
	\else\ifnum\pdfstrcmp{\thechapter}{N}=0% teste si les 2 strings sont indentiques
		colorboxchap%
	\else%
		color16_126_179%
	\fi\fi\fi\fi\fi\fi\fi\fi\fi\fi\fi\fi\fi\fi
}	


\newcommand{\textforonglet}{%
	\ifnum\pdfstrcmp{\thechapter}{0}=0% teste si les 2 strings sont indentiques
		I% % pour introduction
	\else\ifnum\pdfstrcmp{\thechapter}{6}=0% teste si les 2 strings sont indentiques
		C% pour conclusion (vérifier que c'est le 6e chapitre)
	\else%
		\thechapter%
	\fi\fi
}

\newlength{\ongletvtop}
\setlength{\ongletvtop}{25cm}

\newlength{\ongletvpos}

\newlength{\frameboxsize}

\newlength{\espacepaire}

\newcommand{\ongletcote}{%
	% Fixer \ongletheight, \ongletvpos et \frameboxsize
	%%% Traiter manuellement les annexes constituées d'un insert de pdf car pb taille boîtes (bug ?) %%%
	\ifnum\pdfstrcmp{\thechapter}{D}=0% teste si les 2 strings sont indentiques
		\setlength{\ongletheight}{1.7cm}
		\setlength{\frameboxsize}{1.5cm}
		\setlength{\espacepaire}{0.1cm}
		\setlength{\ongletvpos}{\ongletvtop-\ongletheight*\real{\valueforonglet}*\real{1.19}}%
	\else\ifnum\pdfstrcmp{\thechapter}{E}=0% teste si les 2 strings sont indentiques
		\setlength{\ongletheight}{1.7cm}
		\setlength{\frameboxsize}{1.5cm}
		\setlength{\espacepaire}{0.1cm}
		\setlength{\ongletvpos}{\ongletvtop-\ongletheight*\real{\valueforonglet}*\real{1.19}}%
	\else%
		\setlength{\ongletheight}{1.5cm}% environ 1.7cm apparent sur PDF A4 (colorbox déborbe framebox)
		\setlength{\frameboxsize}{1.25cm}
		\setlength{\espacepaire}{0.cm}
		\setlength{\ongletvpos}{\ongletvtop-\ongletheight*\real{\valueforonglet}*\real{1.2}}%
	\fi\fi	%
	% Boîtes
	\ifthenelse{\isodd{\value{page}}}{% IF (Boîte page impaire)
%		\begin{minipage}{3cm}
		\makebox[0pt][l]{%
			\hspace{0.42cm}% petit espace étrange
			\hspace{1cm}% margin outer (2.5cm) - 1.5cm = 1cm (bien centré)
			\raisebox{\ongletvpos}[0pt][0pt]{%
				\colorbox{\coulorforonglet}{%
					\parbox[t][\ongletheight][c]{\ongletwidth}{%
						\setlength{\fboxrule}{0pt}% Masquer trait de framebox
						\framebox[\frameboxsize]{%
							\ongletfont\textforonglet%
						}
					}
				}
			}
		}
%		\end{minipage}
	}{% ELSE (Boîte page paire)
		\makebox[0pt][r]{%
			\raisebox{\ongletvpos}[0pt][0pt]{%
				\colorbox{\coulorforonglet}{%
					\parbox[t][\ongletheight][c]{\ongletwidth}{%
						\setlength{\fboxrule}{0pt}% Masquer trait de framebox
						\framebox[1.25cm]{%
							% ne rien écrire
						}
					}
				}
			}
			\hspace{\espacepaire}% petit espace étrange
			\hspace{1cm}% margin outer (2.5cm) - 1.5cm = 1cm (bien centré)
		}
	}
}

%----------------------------------------------------------------------------------------
%	TITRE SOUS ONGLETS
%----------------------------------------------------------------------------------------

\newcommand{\titreongletfont}{%
	\fontfamily{pag}\fontseries{b}\fontshape{n}\fontsize{0.5cm}{0.6cm}\selectfont\textcolor{\coulorforonglet}%
}

\renewcommand{\chaptermark}[1]{\markboth{\MakeUppercase{#1}}{}}

\newlength{\titreongletvpos}

\newcommand{\titreongletcote}{%
	% Fixer \titreongletvpos
	%%% Traiter manuellement les annexes constituées d'un insert de pdf car pb taille boîtes (bug ?) %%%
	\ifnum\pdfstrcmp{\thechapter}{D}=0% teste si les 2 strings sont indentiques
		\setlength{\titreongletvpos}{\ongletvtop-\ongletheight*\real{\valueforonglet}*\real{1.19}-\ongletheight*\real{1.19}}%
	\else\ifnum\pdfstrcmp{\thechapter}{E}=0% teste si les 2 strings sont indentiques
		\setlength{\titreongletvpos}{\ongletvtop-\ongletheight*\real{\valueforonglet}*\real{1.19}-\ongletheight*\real{1.19}}%
	\else%
		\setlength{\titreongletvpos}{\ongletvtop-\ongletheight*\real{\valueforonglet}*\real{1.2}-\ongletheight*\real{1.2}}%
	\fi\fi%
	% Boîtes
	\ifthenelse{\isodd{\value{page}}}{% IF (Boîte page impaire)
		\makebox[0pt][l]{%
			\hspace{0cm}% (bien centré sur bord texte avant décallage = 0cm)
			\hspace{1.75cm}% margin outer (2.5cm) - 1.5cm = 1cm + moitié boite (0.75cm)
			\raisebox{\titreongletvpos}[0pt][0pt]{\rotatebox{270}{\titreongletfont\leftmark}}
		}
	}{% ELSE (Boîte page paire)
		% ne rien faire
	}
}	

