%%%%%%%%%%%%%%%%%%%%%%%%%%%%%%%%%%%%%%%%%%%%%%%%%%%%%%%%%%%%%%%%
\chapter*{Conclusion}
\addcontentsline{toc}{chapter}{Conclusion}


Mon travail de thèse s'inscrit dans le développement de l'instrument FIRST qui implémente le concept de masquage de pupille fibré pour l'imagerie haut contraste et à haute résolution angulaire (HRA). Il a déjà été montré que FIRST permettait la détection de compagnons substellaires à des séparations inférieures à la limite de diffraction du télescope, ce qui est un avantage considérable pour l'imagerie directe d'exoplanètes avec les télescopes de la classe des $10 \,$m de diamètre. FIRST est intégré au banc d'optique adaptative extrême SCExAO sur le télescope de $8 \,$m de diamètre Subaru. Ma thèse a permis les tests et la caractérisation de composants d'optique intégrée sur sa réplique modifiée en laboratoire (FIRSTv2) dans l'objectif d'augmenter les performances en contraste de l'instrument.

Comme je l'ai présenté dans le premier chapitre de ce manuscrit, les systèmes protoplanétaires sont des cas d'études cruciaux pour notre compréhension de la formation planétaire. Il semblerait que ces systèmes présentent une raie d'émission \ha~provenant du phénomène d'accrétion de matière, diminuant le contraste du système dans ces longueurs d'onde. Le calcul de la phase différentielle, une observable auto-étalonnée des perturbations atmosphériques et instrumentales, est particulièrement adaptée aux systèmes avec un tel spectre.

J'ai ensuite présenté l'installation de FIRSTv2 au laboratoire dans le second chapitre. Les résultats de la caractérisation des deux composants d'optique intégré sur lesquels j'ai travaillé y sont présentés et ont fait l'objet d'un article de conférence. Les transmissions des deux composants ont été estimés à $15\%$ et $30\%$. Je présente aussi les autres composants et discute d'une possibilité de retirer les lignes à retard dans le futur car elles sont trop peu transmissives. J'expose une autre partie de ma thèse qui a consisté au développement du logiciel de contrôle du banc de test, que j'ai ensuite déployé sur ciel. Ce logiciel permet le contrôle synchronisé des différents composants clés (miroir déformable, lignes à retard et caméra) pour la mesure des interférogrammes en sortie de la puce photonique.

Un autre aspect de ma thèse a été le développement du programme de traitement et d'analyse de données. Comme je l'ai décrit dans le troisième chapitre, il s'agit d'étalonner les interférogrammes imagés sur la caméra de l'instrument par le calcul de la P2VM. Celle-ci permet d'ajuster les franges mesurés afin d'estimer la phase des visibilités complexes des bases. À partir de celle-ci est calculée la phase différentielle, qui tire profit du fait qu'on mesure la phase en fonction de la longueur d'onde. Je présente également l'étalonnage spectrale nécessaire à l'analyse des données ainsi que la séquence de modulation temporelle des franges.

Dans le quatrième chapitre je présente le système simulant une source protoplanétaire que j'ai intégré sur le banc de test afin de caractériser l'instrument lors d'observations de ces sources. Cela m'a permis de montrer que les phases différentielles permettent de détecter un compagnon de type protoplanétaire à une séparation de $0,68 \lambda / B$. On a pu voir que les mesures étaient limitées par une modulation de la phase, qu'on appelle \wiggles. L'amplitude de ceux-ci limite la détection de systèmes protoplanétaire à un contraste de $\sim 0,56$. Les efforts sont actuellement investis dans la recherche de la source de cette perturbation ainsi que sa diminution, car on s'attend à des contrastes de l'ordre de $\sim 10^{-2} - 10^{-3}$ dans le cadre d'observations de protoplanètes.

Pour finir, j'ai présenté dans le cinquième chapitre les différentes étapes de l'installation de FIRSTv2 sur la plateforme SCExAO au télescope Subaru qui ont mené à sa première lumière le 10 septembre 2021. J'ai pu intégrer l'instrument et le logiciel de contrôle développé au laboratoire lors de deux missions à Hawaï ainsi qu'à distance avec l'aide précieuse de Sébastien Vievard sur place. Nous avons dû faire face à de nouvelles perturbations sur les mesures des interférogrammes. En effet, la phase est très peu stable au cours du temps et nous pensons que les fibres et la méthode d'acquisition des interférogrammes par la modulation temporelle des franges sont encore trop sensibles aux perturbations présentes sur le banc (vents, vibrations, résidus en aval de l'optique adaptative).

Ainsi, ma thèse a pu montrer la capacité des phases différentielles à détecter un système de type protoplanétaire. Les prochains développements se concentreront sur de nouveaux concepts de puces photoniques (3D, ABCD, haut contraste d'indice, lanterne photonique) qui permettront d'augmenter la sensibilité de l'instrument. De nouveaux efforts sont aussi nécessaires afin de comprendre l'origine des \wiggles~et de les réduire car ils limitent la précision sur les mesures de phase à un contraste du système protoplanétaire observé de $\sim 0,56$. Enfin, un gain en stabilité sur les mesures de phase sur le banc SCExAO est encore nécessaire et deux pistes principales s'offrent à nous : la protection des fibres optiques sur le banc de recombinaison et la suppression de l'acquisition des interférogrammes par la modulation temporelle des franges.

